\subsection*{Genetic relatedness through the lens of tree sequences}

Genetic relatedness is a central concept in genetics, with this year marking the 100th anniversary of the seminal paper by Sewall Wright on `Coefficients of inbreeding and relationship'. 
Genetic relatedness between individuals, or between populations, is of interest in its own right in population and quantitative genetics and their applications to human, animal, or plant settings.
It also plays a crucial role in many downstream analyses, such as principal components analysis (PCA) in population genetics and phenotype prediction in quantitative genetics.
Despite its importance, however, there is no single, agreed-upon measure of genetic relatedness, with numerous efforts having been made to first characterise the concept and subsequently estimate it from the available data.
At its most general, genetic relatedness refers to some notion of similarity between individuals, where similarity can be defined according to pedigree, genotype, or genealogy.
In this poster, we attempt to harmonise different notions of genetic relatedness via the tree sequence encoding of 
Ancestral Recombination Graphs.
A tree sequence represents the relationships between a set of DNA sequences and provides an efficient way of storing genetic variation data.
It also provides a natural way to describe genetic relatedness via the ancestral relationships encoded in the tree sequence.
We demonstrate this using simple examples, and illustrate how to perform common analyses such as PCA directly on the tree sequence.

\clearpage