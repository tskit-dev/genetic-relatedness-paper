\documentclass[11pt,a4paper]{article}
\usepackage[margin=1in]{geometry}
\usepackage[utf8]{inputenc}
\usepackage{authblk}
\usepackage[english]{babel}
\usepackage{amsmath}
\usepackage{amsfonts}
\usepackage{amssymb}
\usepackage{amsthm}
\usepackage{bbm}
\usepackage{mathabx} % for \ldbrack, \rdbrack
\usepackage{graphicx}
\usepackage{natbib}
\bibliographystyle{abbrvnat}
\usepackage{nicefrac}
\usepackage{todonotes}
\usepackage{lineno}
\linenumbers
\usepackage{comment}
\usepackage[acronym]{glossaries}
\usepackage{xcolor}
\usepackage{url}
\usepackage{hyperref}
\hypersetup{
    colorlinks=true,
    linkcolor=blue,
    filecolor=magenta,
    urlcolor=blue,
    citecolor=black
}
\usepackage[most]{tcolorbox}
\usepackage{nicefrac}

%%% MATHS COMMANDS %%%
\newtheorem{lemma}{Lemma}
\newtheorem{definition}{Definition}
\newcommand{\Cov}{\mathbb{C}\text{ov}}
\newcommand{\Cor}{\mathbb{C}\text{or}}
\newcommand{\Var}{\mathbb{V}\text{ar}}
\newcommand{\Prob}{\text{Pr}}
\newcommand{\N}{\mathcal{N}}
\renewcommand{\P}{\mathbb{P}}
\newcommand{\E}{\mathbb{E}}
\newcommand{\ident}[2]{\ldbrack#1=#2\rdbrack}

\newcommand{\bsymb}[1]{\boldsymbol{#1}}

\newcommand{\ppath}{\mathcal{P}}
\newcommand{\sibs}{\mathcal{S}}
\newcommand{\nodes}{\mathcal{N}}
\newcommand{\leT}{\le_T}
\newcommand{\geT}{\ge_T}
\newcommand{\nleT}{\nleq_T}
\newcommand{\pluseq}{\mathrel{+}=}
\newcommand{\minuseq}{\mathrel{-}=}
\newcommand{\timeseq}{\mathrel{*}=}
\newcommand{\diveq}{\mathrel{/}=}

\setlength{\marginparwidth}{2cm}

%%% COMMENTS %%%
\newcommand{\brieuc}[1]{{\textcolor{purple}{{\bf Brieuc:} #1}}}
\newcommand{\peter}[1]{{\textcolor{orange}{{\bf Peter:} #1}}}
\newcommand{\jerome}[1]{{\textcolor{green}{{\bf Jerome:} #1}}}
\newcommand{\gregor}[1]{{\textcolor{blue}{{\bf Gregor:} #1}}}
\newcommand{\georgia}[1]{{\textcolor{blue}{{\bf Georgia:} #1}}}
\newcommand{\luke}[1]{{\textcolor{cyan}{{\bf Luke:} #1}}}

%%% ACRONYMS and things %%%
\newcommand{\eGRM}{\texttt{eGRM}}
\newcommand{\tskit}{\texttt{tskit}}
\newcommand{\scipy}{\texttt{scipy}}
\newcommand{\eigh}{\texttt{eigh}}
\newcommand{\eigsh}{\texttt{eigsh}}
\newcommand{\scikitallel}{\texttt{scikit-allel}}
\newcommand{\ARGneedlelib}{\texttt{ARG-needle-lib}}
\newcommand{\tsGR}{\texttt{ts.genetic\_relatedness}}
\newcommand{\tsGRM}{\texttt{ts.genetic\_relatedness\_matrix}}
\newcommand{\tsGRMw}{\texttt{ts.genetic\_relatedness\_weighted}}
\newcommand{\tsGRMv}{\texttt{ts.genetic\_relatedness\_vector}}
\newcommand{\tsPCA}{\texttt{ts.pca}}
\newcommand{\msprime}{\texttt{msprime}}

\newacronym{grm}{GRM}{genetic relatedness matrix}
\newacronym{arg}{ARG}{ancestral recombination graph}

% These macros are borrowed from TAOCPMAC.tex
\newcommand{\slug}{\hbox{\kern1.5pt\vrule width2.5pt height6pt depth1.5pt\kern1.5pt}}
\def\xskip{\hskip 7pt plus 3pt minus 4pt}
\newdimen\algindent
\newif\ifitempar \itempartrue % normally true unless briefly set false
\def\algindentset#1{\setbox0\hbox{{\bf #1.\kern.25em}}\algindent=\wd0\relax}
\def\algbegin #1 #2{\algindentset{#21}\alg #1 #2} % when steps all have 1 digit
\def\aalgbegin #1 #2{\algindentset{#211}\alg #1 #2} % when 10 or more steps
\def\alg#1(#2). {\medbreak % Usage: \algbegin Algorithm A (algname). This...
  \noindent{\bf#1}({\it#2\/}).\xskip\ignorespaces}
\def\kalgstep#1.{\ifitempar\smallskip\noindent\else\itempartrue
   \hskip-\parindent\fi
   \hbox to\algindent{\bf\hfil #1.\kern.25em}%
   \hangindent=\algindent\hangafter=1\ignorespaces}

\newcommand{\algstep}[3]{\kalgstep #1 [#2] #3 }
\newenvironment{taocpalg}[3]{%
\vspace{1em}%
\algbegin Algorithm #1. ({#2}). #3 }
{\vspace{1em}}

\newcommand{\algorithmref}[1]{#1}

% This doesn't seem to be working, unusually?
% from http://tex.stackexchange.com/questions/43648/why-doesnt-lineno-number-a-paragraph-when-it-is-followed-by-an-align-equation/55297#55297
\ifcsname{patchAmsMathEnvironmentForLineno}\endcsname
    \newcommand*\patchAmsMathEnvironmentForLineno[1]{%
      \expandafter\let\csname old#1\expandafter\endcsname\csname #1\endcsname
      \expandafter\let\csname oldend#1\expandafter\endcsname\csname end#1\endcsname
      \renewenvironment{#1}%
         {\linenomath\csname old#1\endcsname}%
         {\csname oldend#1\endcsname\endlinenomath}}%
    \newcommand*\patchBothAmsMathEnvironmentsForLineno[1]{%
      \patchAmsMathEnvironmentForLineno{#1}%
      \patchAmsMathEnvironmentForLineno{#1*}}%
    \AtBeginDocument{%
    \patchBothAmsMathEnvironmentsForLineno{equation}%
    \patchBothAmsMathEnvironmentsForLineno{align}%
    \patchBothAmsMathEnvironmentsForLineno{flalign}%
    \patchBothAmsMathEnvironmentsForLineno{alignat}%
    \patchBothAmsMathEnvironmentsForLineno{gather}%
    \patchBothAmsMathEnvironmentsForLineno{multline}%
\fi
