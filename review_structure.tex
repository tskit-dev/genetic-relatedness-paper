% THIS IS JUST TO KEEP SIGHT OF THE OVERALL STRUCTURE WHILE I AM WRITTING ...

% --- PEDIGREE ----------------------------------------------------------------

\subsection{Pedigree-based relatedness}

% START OF FORMAL PEDIGREE RELATEDNESS WORK BY FISHER & WRIGHT
\paragraph{Genetic relatedness was first formalised by \cite{wright1922coefficients},
who introduced the coefficients of relationship and inbreeding in the
context of a pedigree.}

% WHAT IS A PEDIGREE
\paragraph{Pedigree is a directed acyclic graph (DAG).} 

% WRIGHT'S RELATIONSHIP COEFFICIENT
\paragraph{Based on his path analysis of phenotypic values for pedigree
members, \cite{wright1922coefficients} defined
\textit{relationship coefficient between individuals $i$ and $j$ as the
correlation between their genetic values}:}

% WRIGHT'S INBREEDING COEFFICIENT
\paragraph{\cite{wright1922coefficients} defined
\textit{inbreeding coefficient of individual $i$ as the correlation
between values of its genomes (``uniting gametes'')}:}

% EARLY PEDIGREE ALGORITHMS
\paragraph{\cite{wright1922coefficients} showed how to calculate
relatedness coefficients in a general pedigree by tracing all pedigree
lineages between relatives.}

% STAT/QUANT GENETICS WITH PEDIGREES
\paragraph{The work of Fisher and Wright has laid foundations for
statistical modelling in quantitative genetics, particularly through the
use of linear mixed models
\citep{falconer1996introduction, henderson1984applications,
lynch1998genetics, mrode2023linear}.}

% MORE PEDIGREE ALGORITHMS
\paragraph{Algorithms for pedigree-based numerator relationship matrix and
corresponding linear mixed models leverage pedigree DAG.}

% PEDIGREE RELATEDNESS EXTENSIONS
\paragraph{There are many extensions of standard pedigree-based relatedness.}

% GAMETIC RELATEDNESS
\paragraph{Gametic relatedness is defined between the genomes of individuals
\citep{\citep{smith1985efficient, schaeffer1989inverse} and is equivalent to
kinship defined in the next sub-section.}

% GROUP RELATEDNESS
\paragraph{Group relatedness is defined between groups of individuals
\citep{wright1949genetical, jacquard1975inbreeding, cockerham1976group}.}

% LIMITATIONS OF PEDIGREE INFORMATION
\paragraph{However, pedigree information is fundamentally limited for
describing exact genetic relatedness between individuals.}

% --- GENOTYPE - early-years - definitions and still probs on pedigrees -------

\subsection{Genotype-based relatedness\textcolor{red}{~ - early years}}

% START OF FORMAL GENOTYPE RELATEDNESS WORK BY MALECOT - BUT FIRST DEFINE GENOS & ALLELES
\paragraph{With the discovery of the structure and role of DNA,
relatedness was redefined with respect to genotypes of individuals.} 

% START OF FORMAL GENOTYPE RELATEDNESS WORK BY MALECOT
\paragraph{Seminal work on relatedness with respect to genotypes has been
done by \cite{cotterman1940calculus} and
\cite{malecot1948mathematiques, malecot1969mathemathics}.}

% INBREEDING & KINSHIP AS PROBABILITIES OF IBD
\paragraph{Malecot redefined Wright's work based on the
probability of identity of alleles; representing the
notion of random sampling of alleles from individuals.}

% WHILE THESE COEFFICIENTS WERE DEFINED ON GENOS, THERE WAS NO GENO DATA, SO PEDIGREES WERE USED
\paragraph{These genotype-based definitions of relatedness were proposed
before DNA data was available.}

% POSITIVE AND NEGATIVE VALUES FOR THE COEFFICIENTS
\paragraph{While Wright's coefficients are based on covariance and
correlation and can hence be positive or negative,
Malecot's coefficients are based on probabilities and are hence
non-negative.}

% WRIGHT'S STATISTICS
\paragraph{The relative nature of relatedness coefficients is the basis
for describing relatedness across (sub-)populations that can be defined
temporally, geographically, or some other grouping.}

% CONNECTION BETWEEN IBS AND IBD & F-STATS
\paragraph{The relative nature of relatedness coefficients is used to
connect IBS and IBD in specific settings.}

% COEFFICIENTS BETWEEN GENOTYPES, INSTEAD OF JUST ITS ALLELES
\paragraph{While inbreeding and kinship coefficients capture information
about the identity of alleles between individuals, they do not fully
capture information about the identity of their genotypes.}

% --- GENOTYPE - based on actual marker data ----------------------------------

\subsection{Genotype-based relatedness}

% EARLY GENOTYPE-BASED RELATEDNESS ESTIMATION WITH A FEW TO HUNDREDS OF MARKERS
\paragraph{Developments of molecular genetics enabled marker data
generation and estimation of genotype-based relatedness.}

% VARIATION IN RELATEDNESS DUE TO RECOMBINATION - OVERALL AND ALONG GENOME
\paragraph{Increasing number of markers improved accuracy, but also
revealed substantial variation from expected relatedness coefficients
between pairs of individuals - overall as well as along genome regions.}

% MODERN GENOTYPE-BASED RELATEDNESS ESTIMATION WITH MANY GENOME-WIDE MARKERS
\paragraph{Further developments in molecular genetics enabled genome-wide
genotyping with SNP arrays and increasingly whole-genome resequencing.}

% SIMPLE ALLELE MATCHING FOR IBS
\paragraph{The simplest estimator of GRM that measures IBS between
individuals is based on allele matching.}

% LINEAR MODEL BASED ESTIMATORS OF GRM
\paragraph{Some popular estimators of GRM are based on linear
modelling of phenotypic values with genome-wide genotype data.}

% VANRADEN VERSION (THE RATIO OF SUMS ESTIMATOR)
\paragraph{The ratio of sums estimator of GRM from \cite{vanraden2008efficient} is:}

% AMIN/YANG/GCTA VERSION (THE AVERAGE OF RATIOS)
\paragraph{The average of ratios estimator of GRM popularised by
\cite{yang2010common} is:}

% SPEED/LDAK VERSION (THE AVERAGE OF WEIGHTED RATIOS)
\paragraph{Genome-wide markers densely tag genome variation
enabling the capture of unknown causal loci, yet not all genome
regions are equivalently tagged.}

% GRM "extensions"
\paragraph*{There is large number of extensions of the above three
GRM estimators for quantitative genetic analyses in different settings.}

% TODO
\paragraph{Fitting the model \eqref{eqn:fisher_pheno_mat_walpha} with a GRM has
reinvigorated quantitative genetics.}