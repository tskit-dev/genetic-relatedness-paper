Genetic relatedness between individuals (or populations) can mean many things.
%
At its most general, given a set of individuals, genetic relatedness refers
to some notion of \textit{similarity} between individuals, where similarity
can be defined according to pedigree, genotype, or genealogy.
%
The aim of this project is twofold:

\begin{enumerate}
    \item Harmonise the different notions of genetic relatedness through the
    lens of tree sequences, outlining the various use cases.
    \item Develop efficient algorithms to compute genetic relatedness and perform
    operations with genetic relatedness information.
\end{enumerate}

\subsection*{Update: $6^{th}$ April 2022}

A couple of related preprints have come out in the last few months that study
the use of the ARG-based relatedness matrix in PCA \citep{fan2022genealogical}
and GWAS, heritability estimation, and mixed effects modelling \citep{zhang2023biobank}.
%
There is overlap in some of the ideas that we've been exploring but there's still
room for an in-depth treatment of genetic relatedness through the lens of tree sequences.
%
The proposal is to focus on objective 1 for now and put together the more
conceptual piece that we've been discussing.
%
We've got a lot of material already so it's mostly just a matter of knitting
everything together.
%
Some additional work that might be worth exploring:

\begin{itemize}
    \item Comparison of site- and branch-based relatedness matrices from the ``unified genealogy'' tree sequence
    \item Comparison of site- and branch-based PCA from the ``unified genealogy'' tree sequence
    \item Discussion of IBD relatedness (i.e. linking into Georgia's work)
    \item Discussion of `A genealogical interpretation of PCA' \citep{mcvean2009genealogical} in the context of tree sequences
\end{itemize}

\newpage

\subsection*{Where are we? (Efficient implementation - parked for now 06/04/22)}

In terms of implementation in \texttt{tskit}, we have a
\href{https://tskit.dev/tskit/docs/stable/python-api.html#tskit.TreeSequence.genetic_relatedness}{function}
to compute genetic relatedness through the statistics framework.
%
We also have a \todo{Brieuc to add tests and documentation in order to get this merged}
\href{https://github.com/tskit-dev/tskit/pull/1246}{pull request} to add a `linear operator'
function that multiplies the genetic relatedness matrix with an arbitrary vector.
%
A limitation of the current implementation of \texttt{ts.genetic\_relatedness} is that
it is relatively slow, especially when it comes to repeated matrix multiplications as
required in PCA \todo{Brieuc to put together a notebook showing the differences + compare algorithmic complexity}
\todo{Peter to do a few back-of-the-envelope algorithmic complexity calculations}.
%
One possible explanation that we'd like to rule out is that caching is slowing us down
- see  \href{https://github.com/tskit-dev/tskit/pull/1937}{this PR}. \smallskip
\todo[inline]{Jerome to have a quick look at this}

We're currently investigating the use of \href{https://tskit.dev/tskit/docs/stable/python-api.html#tskit.TreeSequence.ibd_segments}{\texttt{ts.ibd\_segments}}
as an alternative, approximate way to compute genetic relatedness.
%
Preliminary results suggest that the two methods are indeed equivalent,
though we'd also like to theoretically confirm the equivalence of the two algorithms.
%
There doesn't seem to be a large gain in terms of computational efficiency,
but there's still a bit of work to do to confirm this, particularly along
the lines of approximate computation.
%
\todo[inline]{Georgia to share function to compute genetic relatedness using \texttt{ts.ibd\_segments}}

The remainder of this document is a work-in-progress on the different notions
of genetic relatedness, along with a detailed description of our motivation for
and current implementation of \texttt{ts.genetic\_relatedness}. \smallskip
