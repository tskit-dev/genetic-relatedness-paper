
% --- Application --------------------------------------------------------

\subsection{French-Canadian pedigree}

% Paragraph on the French-Canadian pedigree
To demonstrate the similarities and differences between pedigree and branch relatedness,
we perform a range of analyses on a subset of an extended pedigree of
French-Canadian individuals from the BALSAC project \citep{vezina2020overview}.
%
Spanning over 400 years, this pedigree is compiled from over 4.5 million parish records across Quebec.
%
In this paper, we restricted our analyses to five regions,
each containing five neighboring parishes (Table~\ref{tab:parishes}).
%
Using the migratory patterns from \citet{andersontrocme2023genes} as a reference,
we identified the five regions composed of historically connected parishes to minimize excessive relatedness
within each sampling unit while also mitigating the risk of de-anonymization.
%
Our data access agreements with BALSAC dictated that we use parish records of more than one hundred years old
(before 1924) to publish their metadata and summary statistics.
%
As a result, the pedigree used in this study contains ascending genealogy for
500 randomly chosen contemporary individuals from each of the five regions,
with individuals sampled across the five selected parishes per region.

% Paragraph on filtering individuals
The sub-pedigree obtained from the selected five regions, each containing five parishes,
consisted of 61,490 individuals, including 2,321 probands.
%
A subset of the probands exhibited a low depth of pedigree,
reflecting an incomplete pedigree.
%
To ensure meaningful comparisons, we computed the maximum pedigree depth
for each individual and derived an average depth metric.
%
A total of 48 probands with an average depth of less than 3 were excluded
due to their shallow pedigree.
%
After this filtering step, a total of 2,273 probands were retained for downstream analyses.

\begin{table}
  \centering
  \caption{Selected regions and parishes from the BALSAC French-Canadian Pedigree}
  \renewcommand{\arraystretch}{1.2}
  \resizebox{\textwidth}{!}{%
    \begin{tabular}{|l|p{10cm}|c|}
      \hline
      Region & Parishes & Approximate location \\
      \hline
      Chaleur Bay & St Michel, St François De Sales, St Georges De Malbaie, St Pierre De Malbaie, and St Joseph & 48.5222, -64.2156 \\
      Batiscan & Ste Geneviève De Batiscan, St Luc De Vincennes, St Narcisse, and St Stanislas & 46.5324, -72.3398 \\
      Chaudière & St Georges, St Benoit Labre, St Philibert, St Come, and St Martin De Tours & 46.1184, -70.6691 \\
      L’Assomption & St Jacques, St Alexis, Ste Marie Salomée, and St Esprit & 45.9483, -73.5702 \\
      Mistassini & St Félicien, St Méthode, Notre Dame De La Dore, St Cyrille, and Ste Lucie & 48.6399, -72.4543 \\
      \hline
    \end{tabular}
  }
  \label{tab:parishes}
\end{table}

% Paragraph on pedigree stats and simulated ARGs
We computed pedigree and branch GRM between individuals of interest.
%
Pedigree GRM was computed after
\citet{lange1992calculation} and \citet{colleau2002indirect}.
%
Branch GRM was calculated from an ARG
obtained with the simulation based on pedigree and ancestry
described in \citet{andersontrocme2023genes}.
%
In short, this simulation uses \msprime{} \citep{baumdicker2022efficient} 
for a backward in time simulation in two stages.
%
First, it samples chromosome inheritance through the fixed pedigree
to obtain an ARG within the pedigree.
%
Second, it further samples chromosome inheritance through the assumed demographic model
to recapitate the ARG from the first stage.
%
We simulated the human chromosome 3 in this study because ... \todo{Brieuc: say something here}.
%
To study the stochastic variation in recombination and coalescence events within the pedigree,
we simulated 100 replicates of an ARG using different random seeds.
%
This approach allowed us to assess the variance in branch GRM
while maintaining consistency with the underlying pedigree.

% Paragraph on pedigree and branch PCA
To explore the overall population structure within the pedigree and the simulated tree sequences,
we performed PCA on the set of 2,273 probands.
%
For pedigree PCA, we first computed the pedigree GRM
among the probands and then eigen-decomposed the GRM using
\texttt{eigh} function from \scipy{} \citep{Virtanen2020SciPy}.
%
For branch PCA, to avoid undue influence of large, low-recombination regions,
we first remapped genomic coordinates from base pairs to genetic distance,
using the HapMap II genetic map provided by \texttt{stdpopsim} \citep{adrion2020stdpopsim}.
\todo[inline]{Brieuc/Hanbin: should this have been done also for branch GRM and hence the
above sentence should be moved above where we mention computation of the branch GRM?
(likely we have no time for this for the initial submission!)}
%
We then used Algorithm~\algorithmref{rPCA} to compute the first six PCs.

% Paragraph on comparing relatedness
To compare pedigree and branch relatedness for specific pairwise relationships,
we focused our attention on a subset of the 2,273 probands.
%
Specifically, we randomly sampled one parish per region and
subsampled at least five siblings, first cousins, second cousins, and third cousins from each parish.
%
We then subsampled additional individuals from each parish
to obtain a total of 50 individuals per parish.
%
With these 250 individuals, we computed the pedigree GRM and a branch GRM
for each of the 100 simulated ARGs.

% Paragraph in one vs many ARGs/chromosomes and code
In results, we either show outputs based on branch GRM for chromosome \textcolor{red}{3?} or
for multiple chromosomes.
\todo{Brieuc: fix chromosome number}
%
The code for performing these analyses and the simulation studies is available at
\url{https://github.com/brieuclehmann/genome_simulations/tree/brieuc}.
\todo{Brieuc: Update URL!?}


% --- Benchmarks ---------------------------------------------------------

\subsection{Benchmark simulations}

We assess the computational efficiency of our implementations for branch GRM and PCA,
with simulations, recording the time for calculations for a range of tree sequences.
%
We simulated the tree sequences with \texttt{msprime} \citep{baumdicker2022efficient},
and varied either the genome sequence or the number of sample nodes (=haploid individuals).

All computations were carried out on a single CPU with 4GB of RAM.

\subsubsection{Branch GRM}

We compared our algorithm~(X) for computing the branch GRM to
the implementation in the \eGRM{} package \citep{fan2022genealogical}.
\todo{Reference equation/algorithm (the one that does something with D?)}
%
The default values for simulations parameters were $10^{7}$ for the genome sequence length,
$2^{10}$ for the number of samples, and effective population size of $10^4$.
%
We then varied
genome sequence length from $(10^{4}, 10^{5}, 10^{6}, 10^{7}, 10^{8})$ and
the number of samples from $(2^7, 2^8, 2^9, 2^{10}, 2^{11}, 2^{12})$,
each one at a time.
%
For each simulation setting, we generated 10 tree sequences with different random seeds and
reported the average time taken to compute the GRM with each implementation.

\subsubsection{Branch PCA}

We assessed our branch PCA Algorithm~\algorithmref{rPCA} against a number of comparators
using \scipy{} \citep{Virtanen2020SciPy} and \scikitallel{} \citep{Miles2024scikit}:
%
\begin{itemize}
    \item Calculating branch GRM using equation~(X) followed by eigenanalysis using \eigh{} function from \scipy{}.
    \todo{Add equation/algorithm (the one that does something with D?)}
    \item Eigenanalysis of branch GRM using \eigsh{} function from \scipy{} using the sparse matrix-vector product Algorithm~\algorithmref{bGRM-vec} as a linear operator.
    \item Calculating genotype GRM using equation~\eqref{eqn:grm} followed by eigenanalysis using \eigh{} function from \scipy{}.
    \item Randomized PCA of genotype matrix using \texttt{randomized\_pca} function from \scikitallel{}.
\end{itemize}
%
We used the same simulations as for the branch GRM computation benchmark,
but explored larger sample sizes, ranging across $(2^8, 2^{10}, 2^{12}, 2^{14}, 2^{16}, 2^{18}, 2^{20})$.
%
For each simulation setting, we generated 10 tree sequences with different random seeds and
reported the average time for PCA with each implementation.

