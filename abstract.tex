Genetic relatedness is a central concept in genetics,
underpinning studies of population and quantitative genetics in human, animal, and plant settings.
%
It is typically stored as a genetic relatedness matrix (GRM),
where each element represents pairwise relatedness between individuals.
%
Beyond traditional GRMs based on pedigree, genotype, phylogeny, and coalescent times,
there has been recent interest in defining GRM based on an ancestral recombination graph (ARG).
%
The ARG-based GRMs were found to better capture the structure of a population and
improve association studies relative to the genotype GRM.
%
Here, we unify the different definition of relatedness via the trait-centric perspective, which naturally leverages branch information in an ARG and provides a definition of branch relatedness and corresponding branch GRM.
%
We explore the relationship between branch relatedness and pedigree relatedness through a case study of French-Canadian individuals that have a known pedigree.
%
Through the tree sequence encoding of an ARG, we then derive an efficient algorithm for computing products between the branch GRM and a general vector, without explicitly forming the branch GRM.
%
This algorithm leverages the sparse encoding of genomes with the tree sequence and hence enables large-scale computations with the branch GRM.
%
We demonstrate the power of this algorithm by developing
a randomized principal components algorithm on tree sequence
that enables the analysis of millions of genomes.
%
All algorithms are implemented in the open source \tskit{} Python package.
%
Taken together, this work consolidates the different notions of relatedness as branch relatedness
and by leveraging the tree sequence encoding of an ARG it provides efficient algorithms
that enable computations with the branch GRM that scale to mega-scale genomic datasets.