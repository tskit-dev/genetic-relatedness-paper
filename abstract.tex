Genetic relatedness is a central concept in genetics,
underpinning studies of population and quantitative genetics in human, animal, and plant settings.
%
It is typically stored as a genetic relatedness matrix (GRM),
whose elements are pairwise relatedness values between individuals.
%
This relatedness has been defined in various contexts
based on pedigree, genotype, phylogeny, coalescent times,
and, recently, ancestral recombination graph (ARG).
%
ARG-based GRMs have been found to better capture the structure of a population and
improve association studies relative to the genotype GRM.
%
However, calculating GRMs and further operations with them is fundamentally challenging
due to inherent quadratic time and space complexity.
%
Here, we first discuss the different definitions of relatedness in a unifying context,
making use of the additive model of a quantitative trait
to provide a definition of ``branch relatedness'' and the corresponding ``branch GRM''.
%
We explore the relationship between branch relatedness and pedigree relatedness through a case study of French-Canadian individuals that have a known pedigree.
%
Through the tree sequence encoding of an ARG, we then derive an efficient algorithm for computing products between the branch GRM and a general vector, without explicitly forming the branch GRM.
%
This algorithm leverages the sparse encoding of genomes with the tree sequence and hence enables large-scale computations with the branch GRM.
%
We demonstrate the power of this algorithm by developing
a randomized principal components algorithm for tree sequences
that easily scales to millions of genomes.
%
All algorithms are implemented in the open source \tskit{} Python package.
%
Taken together, this work consolidates the different notions of relatedness as branch relatedness
and by leveraging the tree sequence encoding of an ARG it provides efficient algorithms
that enable computations with the branch GRM that scale to mega-scale genomic datasets.
