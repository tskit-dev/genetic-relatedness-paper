



\subsection*{Computation: obtaining the entire matrix}

% GENERAL IDEA OF DIVERGENCE COMPUTATION
In this section, we describe an efficient algorithm
for computing the matrix of pairwise divergences from a tree sequence.
%
This will be the ``branch statistic'', so that for haploid sampled genomes
$u$ and $v$, we wish to compute $D(u, v)$, which is the length of the path
leading from $u$ and $v$ back to their MRCA summed across all sites.
%
More formally, suppose that the $k^\text{th}$ tree, $T_k$
extends over a distance $\ell_k$ along the genome,
and that $d_{T_k}(u, v)$ is the distance between nodes $u$ and $v$ in that tree.
%
Then, we'd like to compute:
%
$$ D(u, v) = \sum_k \ell_k d_{T_k}(u, v), $$
%
for every $u$ and $v$.
%
In this section we assume that we want to compute $D(u, v)$ \emph{only}
between \emph{leaves} of the tree.
%
Also note that although we refer to ``the current tree'',
this ``tree'' may in fact be a forest
(that is, the tree may not be connected, and have more than one root).

% EFFICIENT DIVERGENCE MATRIX COMPUTATION - OUTLINE
To compute the divergence for all pairs simultaneously and efficiently,
we will move along the tree sequence,
retaining some information along the way in two places.
%
First, for each node $n$,
the quantity $z(n)$ will represent a contribution to $D$ from the branch above $n$.
%
Concretely, the value of $z(n)$ contributes to divergence
between all pairs of nodes such that one of them is below $n$ and the other is not
in the current tree.
%
Second, for each pair of nodes $m$ and $n$,
the quantity $w(m, n)$ will represent a contribution to $D$ between the subtrees below $m$ and $n$.
%
So, the value of $w(m, n)$ contributes to divergence
between all pairs of nodes such that one is below $m$ and the other is below $n$.
%
This is symmetric, so we will write $w(m, n)$ and $w(n, m)$ for the same quantity.
%
Both $z(n)$ and $w(m, n)$ will be zero if $n$ (or, $m$) is not currently in the tree,
and $w(m, n)$ is zero if $m$ is an ancestor of $n$ or vice-versa.
%
When our algorithm is terminated, we will have $w(u, v) = D(u, v)$ for every pair of sample nodes,
and both $z$ and $w$ zero otherwise.

% EFFICIENT DIVERGENCE MATRIX COMPUTATION - MORE
Given a tree $T$ and some values in $z$ and $w$,
we can compute the divergence $D(u, v)$ implied by the current state as follows.
%
First, let $\nodes$ denote the set of nodes in the tree sequence,
let $p_T(n)$ be the parent of $n$ in the tree (or $\phi$ if $n$ is a root)
and say that $m \leT n$ if $n$ is an ancestor of $m$ in the tree.
%
We say that $m$ is a \emph{child} of $n$ if $p(m) = n$.
%
Now, let $\ppath(u, v)$ denote the set of nodes from $u$ and $v$ up to (but not including) their MRCA:
%
$$
    \ppath(u, v) = \{ n \in \nodes \;:\;
    (u \leT n \text{ and } v \nleT n )
    \text{ or }
    (u \nleT n \text{ and } v \leT n ) .
$$
%
Now, $D(u, v)$ is computed using all contributions from $\ppath(u, v)$:
%
\begin{align} \label{defn:D}
    D(u, v)
    =
    \sum_{n \in \ppath(u, v)} z(n)
    +
    \sum_{m, n \in \ppath(u, v) \,:\, m \neq n } w(m, n) .
\end{align}

% EFFICIENT DIVERGENCE MATRIX COMPUTATION - MORE 2
The contributions to divergence of each tree will be recorded in $z$,
and our goal will be to move these contributions into $w$,
and move entries in $w$ down the tree,
in a slow manner so that we accumulate as much shared divergence as possible.
%
Next, we define some operations on $z$ and $w$
that don't change the internal state (i.e., don't change $D$).

% EFFICIENT DIVERGENCE MATRIX COMPUTATION - MORE 3
First, the operation that moves a contribution from $z(n)$ into $w$.
%
For this, we will need $\sibs(n)$,
which will denote the set of nodes that are \emph{siblings} to the path from $n$ up to the root:
%
$$
    \sibs(u) = \{
        n \in \nodes \; : \;
        u \nleT n \text{ and }
        (
            u \leT p_T(n)
            \text{ or }
            p_T(n) = \phi
        )
    \} .
$$
%
Note that $\sibs(n)$ also includes the roots of any trees not including $n$,
if the current tree is disconnected.

\begin{definition}[\texttt{clear\_edge}]
The operation \texttt{clear\_edge}$(n)$ does:
\begin{enumerate}
    \item For each $c \in \sibs(n)$, add $w(n, c) \pluseq z(n)$.
    \item Set $z(n) = 0$.
\end{enumerate}
\end{definition}

Next, the operation that moves an entry from $w(m, n)$ down towards the leaves.

\begin{definition}[\texttt{push\_down}]
The operation \texttt{push\_down}$(n)$ does,
for each $u$ with $w(n, u) > 0$,
\begin{enumerate}
    \item For each child $c$ of $n$, add $w(c, u) \pluseq w(n, u)$.
    \item Set $w(n, u) = 0$.
\end{enumerate}
\end{definition}

%% This could be a Theorem and then a Proof but not for our target audience, I think?
Here is a proof that neither operation changes $D$
(i.e., that if we compute $D(u, v)$ for any $u$ and $v$ before and after the operations,
we will get the same answer):
the first never changes $D$, while the second does not as long as $n$ is not equal to $u$ or $v$.
%
First, consider \texttt{clear\_edge}$(n)$.
%
This operation changes the expression \eqref{defn:D} only if $n \in \ppath(u, v)$.
%
However, if $n \in \ppath(u, v)$ then there is a unique node $c \in \sibs(n) \cap \ppath(u, v)$
(and note that $c \nleT n$ and $n \nleT c$).
%
So, if we denote by $D$ the value before  \texttt{clear\_edge}$(n)$ and $D'$ after,
and similarly for $w$ and $w'$,
we have that:
%
$$
    D(u, v) - D'(u, v)
    =
    z(n)
    -
    (w(n, c) - w'(n, c))
    =
    z(n) - z(n) = 0 .
$$
%
Next, consider \texttt{push\_down}$(n)$, and assume that $n$ is not $u$ or $v$.
%
If $n \in \ppath(u, v)$ then exactly one child $c$ of $n$ is in $\ppath(u, v)$;
note also that $w(n, c) = 0$.
%
So,
\begin{align*}
    D(u, v) - D'(u, v)
    &=
    \sum_{m \in \ppath(u, v)} w(n, m)
    -
    \sum_{m \in \ppath(u, v)} (w(c, m) - w'(c, m)) \\
    &=
    \sum_{m \in \ppath(u, v)} (w(n, m) - w(n, m)) \\
    &= 0.
\end{align*}
% End proof.

% This could similarly be a theorem.
The purpose of the operations above will be to move around contributions to divergence on the tree
so that when we \emph{change} the tree (i.e., add or remove edges) $D$ will be unchanged.
%
To make this concrete, observe that:
%
\begin{align}
    \text{if } z(m) = 0 \qquad & \text{for all } m \ge_T n, \\
    \text{and } w(m, x) = 0 \qquad & \text{for all } m \ge_T n \text{ and all } x,
\end{align}
%
then \emph{removing the edge above $n$} does not change $D$,
as computed by \eqref{defn:D}.
%
Conversely, the same is true of adding an edge above $n$, as long as we still have a tree.
%
To see this, note that removing the edge above $n$
removes nodes from $\ppath(u, v)$ only if $n \in \ppath(u, v)$,
and if so, removes those nodes in $\ppath(u, v)$ that are above $n$.
%
However, any such node $m$ has $z(m) = w(m, x) = 0$, by assumption.

We introduce one more operation, that makes it possible to remove or add edges
without changing $D$.

\begin{definition}[\texttt{clear\_spine}]
    The operation \texttt{clear\_spine}$(n)$ does:
    let $m_1 \geT m_2 \geT \cdots \geT m_k = n$ be all nodes above $n$, ordered top-down.
    Then, for each $1 \le j \le k$ do \texttt{push\_down}$(m_j)$ and \texttt{clear\_edge}$(m_j)$.
\end{definition}

Note that this works \emph{top-down} so that the ``spine'' (i.e., the path above $n$)
is ``clear'' (i.e., has zero contribution) afterwards.

Finally, we present the algorithm for computing $D$ across a tree sequence.
%
This works by storing contributions from branches in $z$,
and whenever the subtree below a node $n$ changes due to adding or removing of edges below it,
using \texttt{clear\_edge} and \texttt{push\_down} to move contributions from $n$
into lower-down entries in $w$.
%
In addition to $z$ and $w$ we will have two additional bits of bookkeeping:
$q(n)$, which will be the last position along the genome that the subtree below $n$ changed,
and $X$, the current position on the genome.

\begin{taocpalg}{D}{Divergence matrix}
{
    Given a sequence of positions $x_i$ for $1 \le i \le N$ along the genome
    and corresponding sequences of edges to remove ($R_i$) and add ($A_i$) at each position,
    compute the total branch length divergence $D(u, v)$ for any two leaves $u$ and $v$,
    if these are leaves along the entire tree sequence.
    %
    Let $\ell_T(n)$ be the length of the branch above $n$ in $T$ (or zero, if $n$ has no parent),
    initialize $q(n) = 0$ for all $n \in \nodes$ and $i = 1$,
    and at all times, define $z(n) = \ell_T(n) (x_i - q(n))$.
}

\algstep{D0.}{Initialization}{
    Build the first tree by adding all edges in $A_0$.
}

\algstep{D1.}{Remove edges}{
    For each edge $(c, n) \in R_i$, do \texttt{clear\_edge}$(c)$
    and \texttt{clear\_spine}$(n)$,
    Then, 
    set $q(m) = x_i$ for each $m \ge_T c$
    and remove the edge.
}

\algstep{D2.}{Add edges}{
    For each edge $(c, n) \in A_i$, do \texttt{clear\_spine}$(n)$,
    Then, add the edge and
    set $q(m) = x_i$ for each $m \ge_T c$.
}

\algstep{D3.}{Iteration}{
    If $i < N$,
    set $i = i + 1$ and return to \algorithmref{D1}.
}

\end{taocpalg}

Here is a proof that this algorithm computes the total branch length divergence.
%
Let $D_i(u, v)$ be the total branch length divergence up to (and including) the $i^\text{th}$ tree.
%
First, observe that for the first tree, $z(n)$ is equal to the contribution of each edge to divergence,
and so $D$ computed at that point in the algorithm is $D_1$.
%
Next, observe that when we move to the next tree,
since the current position $x_i$ changes, the amount that is added to each $z(n)$ is equal to 
the contribution of the branch above $n$ in that tree.
%
Since the operations above otherwise do not change $D$,
the result follows by induction.
