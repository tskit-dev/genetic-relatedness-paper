
% --- BRIEF INTRO -------------------------------------------------------------

% Paragraph on Background
Relatedness is one of the central concepts in genetics.
%
In its most general sense, genetic relatedness refers to the notion of
similarity between individuals' genomes.
%
These similarities are usually organised as a pairwise comparison of the
genomes within an individual and between individuals, or groups of
individuals.
%
As one of the central genetic concepts, relatedness is used in many
applications \citep{weir2006genetic, speed2015relatedness}.
%
For example, it has been used to describe genetic variation within and between individuals
and groups of individuals in population genetics
\citep{crow2009introduction, charlesworth2010elements},
to analyse phenotype covariation between close and distant relatives in
quantitative genetics \citep{falconer1996introduction, lynch1998genetics},
and to estimate genetic changes in phenotypic variation over time in
evolutionary genetics \citep{walsh2018evolution, arnold2023evolutionary}.
%
Over time, genetic relatedness has been defined according to
pedigree \citep{fisher1919correlation, wright1922coefficients},
genotype \citep{cotterman1940calculus, malecot1948mathematiques, malecot1969mathemathics},
phylogeny \citep{felsenstein1985phylogenies,lynch1991methods},
coalescent times \citep{slatkin1991inbreeding}, and
% moving this two paragraphs down -> let's keep it here to create a link to the next paragraph
% (as in why ARG
recently, ancestral recombination graph \citep{tsambos2022efficient, fan2022genealogical, zhang2023biobank}.
%
% \textcolor{red}{We review the standard definitions in Appendix: A Brief History of Genetic Relatedness}.
%
For a set of individuals, we store their pairwise relatedness in a genetic relatedness matrix (GRM).
    
% Paragraph on ARG and relatedness
Ancestral recombination graphs (ARGs) 
describe the network of inheritance relations between a set of individuals
via the action of recombination and mutation within a (usually implicit) pedigree
% JK: rationale for these 4 refs is that they are all review papers from the past
% year. This covers the various different modern perspectives of what an ARG is.
\citep{brandt2024promise, lewanski2024era, wong2023general, nielsen2024inference},
and so provide a common framework in which to consider
the various concepts of relatedness.
%
Although ARGs are not directly observable,
there has been significant recent progress in inferring ARGs from a sample of DNA sequences
\citep{speidel2019method, kelleher2019inferring, zhang2023biobank, deng2024robust, gunnarsson2024scalable}.
%
This has been accompanied by computational advances that enable
the highly efficient storage and processing of ARGs
\citep{kelleher2016efficient, zhu2024variance, dehaas2024enabling};
in this paper we make use of the \textit{succinct tree sequence} encoding
\citep{ralph2020efficiently, wong2023general}
made available through the \tskit{} library.

% Paragraph on Branch relatedness
In addition to providing a unifying framework,
ARGs have led to new formulations of relatedness.
%
The ``eGRM'' of \citet{fan2022genealogical}
defines the relatedness between two individuals
in terms of the total area of branches in the ARG that are ancestral to both,
similar to previous single-tree definitions \citep{slatkin1991inbreeding}.
%
\citet{fan2022genealogical} showed this is the expected genotype relatedness
under a Poisson model of mutation,
a special case of a more general duality between ``branch''
and ``site'' statistics \citep{ralph2019empirical, ralph2020efficiently}.
%
The same concept was used by \citet{zhang2023biobank},
although with different terminology,
who connected their definition of the ``ARG-GRM'' to
the time to most recent common ancestor (TMRCA)
of a single tree \citep{slatkin1991inbreeding, mcvean2009genealogical}.
%
There are many different notions of relatedness (see Box 1 for a 
brief overview), usually defined as an expectation of some 
quantity (e.g., pedigree relatedness is the expected genetic identity within a pedigree).
We therefore use the more precise terms 
``branch relatedness'' and ``branch GRM'' 
rather than previously proposed ``eGRM'' or ``ARG-GRM'' 
to avoid confusion (see Methods for more discussion).

% Paragraph on Uses of branch GRM
Recent applications of these methods have highlighted the advantages of
using branch information to improve genetic analyses.
%
\citet{fan2022genealogical} demonstrate that the branch GRM (their ``eGRM'')
better describes population structure relative to the corresponding genotype GRM,
even when based on the same genetic information,
and can provide time-resolved characterisations of population structure
by considering shared branch areas on particular subsets of the ARG
defined by specific time intervals.
%
\citet{zhang2023biobank} use a branch GRM (their ``ARG-GRM'')
to estimate heritability and to perform a ``genealogy-wide association scan'',
showing that this approach is more powerful at detecting the effect of rare variants
than association analysis on SNP array genotypes imputed to whole-genome sequence genotypes.
%
\citet{tsambos2022efficient} developed a method
to find DNA segments that are identical-by-descent (IBD) for pairs of individuals in a given ARG and
then summarise these outputs, possibly as an ``IBD GRM'',
which provides an ARG-based analogue to the pedigree GRM.

% --- GAP ---------------------------------------------------------------------

% Paragraph on The gap
However, the scalability of current ARG-based relatedness methods is limited by their need to store the full branch GRM in memory. This restricts their use for downstream tasks such as principal component analysis (PCA) to datasets of between 10,000 and 100,000 individuals, depending on the available computational resources. Such downstream analyses can sometimes be performed instead via matrix-vector operations. The Algorithms that instead rely on matrix-vector operations, that is 
%
%Operationally, \citet{fan2022genealogical} shifted the estimation of relatedness from summing over loci of a genotype matrix and for each locus evaluating genotype similarity between samples (for genotype GRM) to summing over local trees of an ARG and for each local tree evaluating time between samples' MRCA and the root (for branch GRM).
%
%Iterating over local trees can be computationally expensive for large genomic datasets with a huge number of past recombination events, and hence a huge number of local trees.
%
% \peter{I think this is true: Both} \citet{fan2022genealogical} and
% \gregor{Peter: they did it to derive, but not to do computations}
%\citet{zhang2023biobank} also approximate the branch GRM by sampling new mutations on the branches of the ARG and computing the genotype GRM from these mutations, and demonstrate that this approach converges to the true branch GRM with high mutation rates, in line with the branch-site duality \citep{ralph2019empirical, ralph2020efficiently}.
%
%Although these methods have used branch GRM in downstream analyses, such as principal component analysis (PCA), estimation of heritability, and genome-wide associations, each of these approaches requires storing the full branch GRM in memory, limiting analyses between 10,000 and 100,000 individuals (depending on the available computational resources).

% --- AIM ---------------------------------------------------------------------

% Paragraph on The aim
In this paper, we explore the relationships between different notions of relatedness and
describe several algorithms to perform computations of, and with, the branch GRM.
%
Specifically, we show how branch relatedness arises as
the covariance of a trait process evolving along the branches of an ARG,
and how this relates to other measures of relatedness.
%
We describe a procedure to compute the entire branch GRM by efficiently computing TMRCAs
using the Schieber-Vishkin algorithm \citep{Schieber1988On}.
%
We also develop a new algorithm to perform efficient branch GRM-vector product,
which avoids storing the full branch GRM in memory.
%
This facilitates the use of the randomised singular value decomposition \citep{halko2011findingstructure}
for PCA of over a million samples.
%
We end with a comparison of pedigree and branch relatedness
for simulated data from a real pedigree of the French-Canadian population of
\citet{andersontrocme2023genes},
illustrating the variability of the branch GRM within a fixed pedigree.
%
All computations are demonstrated using Python \tskit{} library
\citep{ralph2020efficiently, kelleher2024tskit}.
