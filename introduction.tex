
% --- BRIEF INTRO -------------------------------------------------------------

% Paragraph on Background
Relatedness is a fundamental concept in genetics.
%
In its most general sense, genetic relatedness refers to the notion of
similarity between individuals' genomes.
%
These similarities are usually organised as a pairwise comparison of the
genomes within an individual and between individuals, or groups of
individuals.
%
As one of the central genetic concepts, relatedness is used in many
applications \citep{weir2006genetic, speed2015relatedness}.
%
For example, it has been used to describe genetic variation within and between individuals
and groups of individuals in population genetics
\citep{crow2009introduction, charlesworth2010elements},
to analyse phenotype covariation between close and distant relatives in
quantitative genetics \citep{falconer1996introduction, lynch1998genetics},
and to estimate genetic changes in phenotypic variation over time in
evolutionary genetics \citep{walsh2018evolution, arnold2023evolutionary}.
%
For a set of individuals, we operationally store their pairwise relatedness in
a genetic relatedness matrix (GRM).
%
Over time, genetic relatedness and GRMs have been defined according to
pedigree \citep{fisher1919correlation, wright1922coefficients},
genotype \citep{cotterman1940calculus, malecot1948mathematiques, malecot1969mathemathics},
phylogeny \citep{felsenstein1985phylogenies,lynch1991methods},
coalescent times \citep{slatkin1991inbreeding}, and
% moving ARG two paragraphs down -> let's keep it here to create a link to the next paragraph
% (as in why ARG
recently, ancestral recombination graph \citep{tsambos2022efficient, fan2022genealogical, zhang2023biobank}.
%
% \textcolor{red}{We review the standard definitions in Appendix: A Brief History of Genetic Relatedness}.
    
% Paragraph on ARG and relatedness
Ancestral recombination graphs (ARGs) 
describe the network of inheritance relations between a set of individuals
via the action of recombination and mutation within a (usually implicit) pedigree
% JK: rationale for these 4 refs is that they are all review papers from the past
% year. This covers the various different modern perspectives of what an ARG is.
\citep{brandt2024promise, lewanski2024era, wong2023general, nielsen2024inference},
and so provide a common framework in which to consider
the various concepts of relatedness.
%
Although ARGs are not directly observable,
there has been significant recent progress in inferring ARGs from a sample of DNA sequences
\citep{rasmussen2014genome,speidel2019method, kelleher2019inferring, zhang2023biobank, deng2024robust, gunnarsson2024scalable}.
%
This has been accompanied by computational advances that enable
the highly efficient storage and processing of ARGs
\citep{kelleher2016efficient, zhu2024variance, dehaas2024enabling}.
%
In this paper we make use of the \textit{succinct tree sequence} 
ARG encoding \citep{ralph2020efficiently, wong2023general}
made available through the \tskit{} library.

% Paragraph on Branch relatedness
In addition to providing a unifying framework,
ARGs have led to new formulations of relatedness.
%
The ``eGRM'' of \citet{fan2022genealogical}
defines the relatedness between two individuals
in terms of the total area of branches in the ARG that are ancestral to both,
similar to previous single-tree definitions \citep{slatkin1991inbreeding}.
%
\citet{fan2022genealogical} showed this is the expected genotype relatedness
under a Poisson model of mutation,
a special case of a more general duality between ``branch''
and ``site'' statistics \citep{ralph2019empirical, ralph2020efficiently}.
%
The same concept was used by \citet{zhang2023biobank},
although with different terminology,
who connected their definition of the ``ARG-GRM'' to
the time to most recent common ancestor (TMRCA)
of a single tree \citep{slatkin1991inbreeding, mcvean2009genealogical}.
%
There are many different notions of relatedness (see Box 1 for a 
brief overview), usually defined as an expectation of some 
quantity (e.g., pedigree relatedness is the expected genetic identity within a pedigree).
We therefore use the more precise terms 
``branch relatedness'' and ``branch GRM'' 
rather than previously proposed ``eGRM'' or ``ARG-GRM'' 
to avoid confusion.

% Paragraph on Uses of branch GRM
Recent applications of these methods have highlighted the advantages of
using branch information to improve genetic analyses.
%
\citet{fan2022genealogical} demonstrate that the branch GRM (their ``eGRM'')
better describes population structure relative to the corresponding genotype GRM,
even when based on the same genetic information,
and can provide time-resolved characterisations of population structure
by considering shared branch areas on particular subsets of the ARG
defined by specific time intervals.
%
An application of ``eGRM'' has also improved mapping quantitative trait loci
in the presence of allelic heterogeneity and in understudied populations \citep{link2023tree}.
%
\citet{tsambos2022efficient} developed a method
to find DNA segments that are identical-by-descent (IBD) for pairs of individuals in a given ARG and
then summarise these outputs, possibly as an ``IBD GRM'',
which provides an ARG-based analogue to the pedigree GRM.
%
\citet{zhang2023biobank} use a branch GRM (their ``ARG-GRM'')
to estimate heritability and to perform a ``genealogy-wide association scan'',
showing that this approach is more powerful at detecting the effect of rare variants
than association analysis on SNP array genotypes imputed to whole-genome sequence genotypes.
%
\citet{gunnarsson2024scalable} extended this work to a large whole-genome sequence dataset
and \citet{zhu2024variance} used randomized linear algebra to
scale the estimation of heritability and region-based association testing with branch GRM.

% --- GAP ---------------------------------------------------------------------

% Paragraph on The gap
The scalability of current ARG-based relatedness methods, however, is constrained by
their need to generate and store the full branch GRM. As the GRM encodes all 
pairwise relationships among $n$ samples, it requires at least $O(n^2)$
time and space to compute. 
Several datasets, currently available and of core interest for these 
methods, consist of hundreds of thousands of 
samples~\citep{caulfield2017national,turnbull2018100,
bycroft2018genome,backman2021exome,Ros-Freixedes2022pig_wgs_variants,
halldorsson2022sequences,uk2023whole,all2024genomic}
and genomic datasets with millions of samples are increasingly available
\citep[e.g.][]{Cesarani2022, stark2024call,cook2025our, Cole2025Invited}.
At this scale, algorithms with quadratic time and space complexity are simply
not feasible, and there is a pressing need for more efficient approaches.
It is important to note, however, that the GRM itself is often not the goal;
rather, we are usually interested in what we can do \emph{with} the GRM.
For example, population genetic applications such as principal component analysis (PCA)
and quantitative genetic applications such as estimation of heritability, are
defined in terms of core linear algebra operations performed on the GRM,
and the outputs are of much smaller dimension. Given that all the information
in a GRM is encoded in an ARG, there is the possibility that we can bypass
generating large intermediate matrices and instead compute the quantities
of interest directly.

% to store the full branch GRM in memory, limiting downstream analyses
% such as principal component analysis (PCA) to datasets of 10,000 to 100,000
% individuals, depending on available computational resources. However, many of
% these analyses can be performed using linear algebra techniques that rely
% solely on matrix-vector operations. That is, these algorithms require only the
% product of the matrix with an arbitrary vector, eliminating the need to store
% the full matrix. 

% Efficient algorithms for computing such matrix-vector products
% can therefore significantly enhance the scalability of downstream analyses.
% \todo[inline]{Add a bit more specificity around how PCA can be performed using
% matrix-vector operations, or examples of other downstream tasks that can be
% done with matrix-vector operations?}

%Operationally, \citet{fan2022genealogical} shifted the estimation of relatedness from summing over loci of a genotype matrix and for each locus evaluating genotype similarity between samples (for genotype GRM) to summing over local trees of an ARG and for each local tree evaluating time between samples' MRCA and the root (for branch GRM).
%
%Iterating over local trees can be computationally expensive for large genomic datasets with a huge number of past recombination events, and hence a huge number of local trees.
%
% \peter{I think this is true: Both} \citet{fan2022genealogical} and
% \gregor{Peter: they did it to derive, but not to do computations}
%\citet{zhang2023biobank} also approximate the branch GRM by sampling new mutations on the branches of the ARG and computing the genotype GRM from these mutations, and demonstrate that this approach converges to the true branch GRM with high mutation rates, in line with the branch-site duality \citep{ralph2019empirical, ralph2020efficiently}.
%
%Although these methods have used branch GRM in downstream analyses, such as principal component analysis (PCA), estimation of heritability, and genome-wide associations, each of these approaches requires storing the full branch GRM in memory, limiting analyses between 10,000 and 100,000 individuals (depending on the available computational resources).

% --- AIM ---------------------------------------------------------------------

% % Paragraph on The aim
% In this paper, we first explore the relationships between different 
% concepts of relatedness, and show how these are unified 

% and define an efficient algorithm
% describe several algorithms to perform computations of, and with, the branch GRM.
% %
% Specifically, we show how branch relatedness arises as
% the covariance of a trait process evolving along the branches of an ARG, and how this relates to other measures of relatedness.
% %
% We undertake a comparison of pedigree and branch relatedness
% for simulated data from a real pedigree of the French-Canadian population of
% \citet{andersontrocme2023genes},
% illustrating the variability of the branch GRM within a fixed pedigree.
% %
% % We describe a procedure to compute the entire branch GRM by efficiently computing TMRCAs
% % using the Schieber-Vishkin algorithm \citep{Schieber1988On}.
% %
% We also develop a new algorithm to perform efficient branch GRM-vector product,
% which avoids storing the full branch GRM in memory.
% %
% This facilitates the use of the randomised singular value decomposition \citep{halko2011findingstructure}
% for PCA of over a million samples.
% %
% All computations are demonstrated using Python \tskit{} library
% \citep{ralph2020efficiently, kelleher2024tskit}.

We begin by defining a trait-centric concept of genetic relatedness, following
long-standing approaches in the 
field~\citep{fisher1919correlation, wright1922coefficients}. 
We show how branch relatedness arises as the
covariance of a trait process evolving along the branches of an ARG, and how
this relates to other measures of relatedness.
% The general strategy is to consider a hypothetical trait where the genetic
% value of an individual can be modelled as a random variable that is a
% function of the individual's alleles arising from past mutations on branches
% encoded with the ARG and the effect of these alleles. The covariance
% structure of this trait between individuals in a given population provides a
% measure of relatedness, and we show how this unifies several common concepts
% of relatedness. 
We then illustrate these definitions and the relationships between pedigree
relatedness and branch relatedness using simulated data from a real pedigree of
French-Canadian individuals. 
As discussed in the previous paragraph, calculations requiring an explicit 
GRM are limited in scale, and we show that the time complexity of 
computing a branch GRM is 
% JK: I know this is different notation to the rest of the paper, but 
% I find the n_t n_s stuff a bit tedious and lacking clarity (totally 
% get the necessity later on, though). Feel free to change if people think
% this will be confusing.
$O(t n^2)$, where $t$ is the number of trees (or equivalently, number 
of recombination breakpoints) in the ARG.
We present an algorithm to compute the product of the branch GRM with
an arbitrary vector, and show that it has 
$O(n + t \log{n})$ time complexity and $O(n)$ space complexity.
We can therefore compute branch GRM-vector products 
\emph{substantially faster} and with less memory 
than the branch GRM itself.
We illustrate the utility of this approach by presenting a
randomised singular value decomposition method 
for PCA on the branch GRM that
uses the new branch GRM-vector product algorithm (implemented 
in \tskit{}), and show that it scales to millions of samples
via benchmarks.
